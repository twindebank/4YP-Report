\section{Conclusion}
\label{sec:conc-conc}
    Two classification models have been built that take the premise of a mosquito detector closer to becoming a reality. By formalising a pipeline and building a software utility around the pipeline, the objectives set at the beginning of this project have been achieved, where specifications are narrowly breached in terms of the allowable rejection ratio criteria. There are further improvements to be made throughout the pipeline at each stage. Examples include more advanced audio normalisation, more exploration of feature-space design in regards to parameter optimisation, an increased number of classifiers and more sophisticated aggregation.
    
    Detailed care has been taken at each stage of building the pipeline, with a wide assortment of both existing and new methods applied. In particular, the concept of asymmetrical rejection is introduced and shown to have significant benefits and control over rejecting symmetrically. 

    These findings were made possible through the use of a utility coded along side the project. Usage of this code will go beyond this project and will likely aid in future research in this field. 
