\section{Future Work}
\label{sec:conc-future}
    \subsection{Model Development}
    \label{subsec:conc-future-dev}
        Although thorough model development has taken place throughout this project, there are many areas in which more time and resources could be devoted to. Potential extensions are highlighted where relevant in each section. Should interest exist, then research could carry on and the model could be tuned further. 
    \subsection{Public Release}
    \label{subsec:conc-future-public}
        The code-base built for this report will be made public under an open-source license, along side some sample data. This, in line with the projects general motivate, will aid in the end-goal of providing a utility for the prevention and reduction of mosquito-borne disease.

    \subsection{Other Datasets}
    \label{subsec:conc-future-datasets}
        The software framework built for this project is capable of more than mosquito detection. With some minor refactoring, the MozzPy package could be used outside the bioacoustics domain, for any binary classification problem, providing invaluable resources for tackling problems of this nature. 
        
    \subsection{Practical Implementations}
    \label{subsec:conc-future-prac}
        To move the HumBug project further towards its goals, practical implementation is necessary; example platforms include Android/iOS smartphones and dedicated hardware. Translating the work detailed in this report to a real-time system comes with many challenges and is no simple task. However, the findings of this report as well as MozzPy as a research tool will potentially aid real-time, accurate, mosquito detection in becoming a realisation.
    