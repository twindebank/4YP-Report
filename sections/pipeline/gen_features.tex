\section{Generating Features}
\label{sec:pl-feats}
    \subsection{Feature Definition}
    \label{subsec:pl-feats-def}
        When using traditional classifiers, it is important that the input data for training and testing classifiers is descriptive of the classification problem being solved. The success of traditional machine learning algorithms depends upon the predictive attributes of the input features; a feature being something that describes a measurable property of a phenomenon being observed \cite{Bishop2006}. Given a time-series signal, a large set of features may be generated to describe and highlight characteristics of the data. The Culex \textit{q.} dataset is comprised of audio signals and labels only, so features must be extracted from the audio signals; these are derived from the signal, so are termed as \textit{intrinsic} features. \textit{Extrinsic} features can also be collected, such as the age of mosquito and location of recording. These would be trivial to collect in a smartphone implementation and could potentially be informative towards classification; however, this information is unavailable for the provided datasets.
        
    \subsection{Audio Features}
    \label{subsec:pl-feats-audio}
        \begin{table}[ht]
            \ssmall
            \singlespacing
            \centering
            % \scalebox{0.8}{
                \begin{tabular}{|l|l|l|c|c|c|} 
                    \hline
                    Name & Description & Default Parameters & Type & Dim. & \specialcell{Code\\Source}\\
                    \hline
                    \hline
                        \specialcell{
                            Hertz-Magnitude\\ Spectrum
                        }&\specialcell{
                            The magnitude of a frequency-domain signal\\
                            representation based on an efficient implementation of \\
                            the Discrete Fourier Transform (DFT), the Fast Fourier \\
                            Transform (FFT) \cite{Cochran1967}.\\
                        }&\specialcell{
                            \code{winlen} $=0.025$: size of feature \\windows (s)\\
                            \code{winstep} $=0.01$: step size between \\windows (s)\\
                            \code{nfft} $=512$: length of FFT to use
                        }&\specialcell{
                            Spectral
                        }&\specialcell{
                            $\mathbb{R}^{257}$
                        }&\specialcell{
                            \cite{Lyons}
                        }\\
                    \hline
                        \specialcell{
                            Mel-Magnitude\\
                            Spectrum
                        }&\specialcell{
                            Frequency-domain signal representation corresponding\\
                            to the subjective human perception of pitch \cite{Stevens1937}.\\
                            The hertz spectrum is mapped to the\\
                            mel-spectrum using the (non-unique) relationship\\
                            $m(f) = 2595\log_{10}(1+f/700)$ \cite{OShaughnessy1987}.
                        }&\specialcell{
                            \code{low\_frq} $=0$: lower frequency bound \\(Hz)\\
                            \code{n\_flt} $=26$: number of mel filterbank \\fitlers\\
                            \code{delta} $=0$: spectral slicing parameter\\
                            \code{a\_min} $=-100$: limit for min amplitude\\
                            \code{a\_max} $=100$: limit for max amplitude\\
                            \code{winlen} $=0.025$\\
                            \code{winstep} $=0.01$
                        }&\specialcell{
                            Spectral
                        }&\specialcell{
                            $\mathbb{R}^{26}$
                        }&\specialcell{
                            \cite{Zilli2016,Lyons}
                        }\\
                    \hline
                        \specialcell{
                            Mel-Frequency\\
                            Cepstral\\Coefficients
                        }&\specialcell{
                            Mel-scale frequencies converted back into a pseudo-time \\
                            domain (cepstrum) by taking the discrete cosine transform\\
                            of the logarithm of each mel-frequency, where amplitudes\\
                            are taken as coefficients.
                        }&\specialcell{
                            \code{nmfccs} $=13$: number of coefficients\\ to generate\\
                            \code{a\_min} $=-100$\\
                            \code{a\_max} $=100$\\
                            \code{winlen} $=0.025$\\
                            \code{winstep} $=0.01$
                        }&\specialcell{
                            Spectral
                        }&\specialcell{
                            $\mathbb{R}^{13}$
                        }&\specialcell{
                            \cite{Zilli2016,Lyons}
                        }\\
                    \hline
                        \specialcell{
                            Zero-Crossing\\
                            Rate
                        }&\specialcell{
                            Rate at which the raw audio signal changes sign:\\
                            $zcr = \frac{1}{T-1}\sum^{T-1}_{t=1}1_{\mathbb{R}_{<0}}(s_ts_{t-1})$ \cite{Chen1988}.
                        }&\specialcell{
                            \code{winlen} $=0.025$\\
                            \code{winstep} $=0.01$
                        }&\specialcell{
                            Temporal
                        }&\specialcell{
                            $\mathbb{R}^{1}$
                        }&\specialcell{
                            \cite{Giannakopoulos2015}
                        }\\
                    \hline
                        \specialcell{
                            Energy
                        }&\specialcell{
                            Window-normalised sum of squares of raw signal values:\\
                            $E=\frac{1}{T}\sum_{t=0}^{T-1}s_t^{2}$.
                        }&\specialcell{
                            \code{winlen} $=0.025$\\
                            \code{winstep} $=0.01$
                        }&\specialcell{
                            Temporal
                        }&\specialcell{
                            $\mathbb{R}^{1}$
                        }&\specialcell{
                            \cite{Giannakopoulos2015}
                        }\\
                    \hline
                        \specialcell{
                            Entropy of\\
                            Energy
                        }&\specialcell{
                            Measure of abrupt changes in signal energy over \\
                            sub-windows: $H_E = -\sum_{\text{sw}}E_{\text{sw}}\times\log_2{E_{\text{sw}}}$
                        }&\specialcell{
                            \code{n\_subwins} $=10$\\
                            \code{winlen} $=0.025$\\
                            \code{winstep} $=0.01$
                        }&\specialcell{
                            Temporal
                        }&\specialcell{
                            $\mathbb{R}^{1}$
                        }&\specialcell{
                            \cite{Giannakopoulos2015}
                        }\\
                    \hline
                        \specialcell{
                            Spectral Centroid\\
                            and Spread
                        }&\specialcell{
                            The spectral centroid characterises the `centre of mass'\\
                            of the spectrum by taking the mean of the frequencies\\
                            weighted by their magnitudes: $SC = \frac{\sum_{f}fS(f)}{\sum_{f}S(f)}$ \cite{Grey1978}.\\
                            The spectral spread is a measure of signal bandwidth: \\
                            $SS = \frac{\sum_{f}(f - SC)^2S(f)}{\sum_{f}S(f)}$ \cite{Grey1978}.
                        }&\specialcell{
                            \code{winlen} $=0.025$\\
                            \code{winstep} $=0.01$
                        }&\specialcell{
                            Spectral
                        }&\specialcell{
                            $\mathbb{R}^{2}$
                        }&\specialcell{
                            \cite{Giannakopoulos2015}
                        }\\
                    \hline
                        \specialcell{
                            Spectral Entropy
                        }&\specialcell{
                            Entropy of the normalised spectral energies for a set of \\
                            sub-frames: $H_S = -\sum_{\text{sw}}\frac{S_{\text{sw}}^2}{\sum_{f}S(f)^2}\times\log_2{\frac{S_{\text{sw}}^2}{\sum_{f}S(f)^2}}$
                        }&\specialcell{
                            \code{n\_subwins} $=10$\\
                            \code{winlen} $=0.025$\\
                            \code{winstep} $=0.01$
                        }&\specialcell{
                            Spectral
                        }&\specialcell{
                            $\mathbb{R}^{1}$
                        }&\specialcell{
                            \cite{Giannakopoulos2015}
                        }\\
                    \hline
                        \specialcell{
                            Spectral Flux
                        }&\specialcell{
                            Squared difference of the normalised spectra magnitudes \\
                            for two successive frames: \\
                            $\phi = \sum_{f}(\frac{S_w(f)}{\sum_{f`}S_w(f`)} - \frac{S_{w-1}(f)}{\sum_{f`}S_{w-1}(f`)})^2 $
                        }&\specialcell{
                            \code{winlen} $=0.025$\\
                            \code{winstep} $=0.01$\\
                        }&\specialcell{
                            Spectral
                        }&\specialcell{
                            $\mathbb{R}^{1}$
                        }&\specialcell{
                            \cite{Giannakopoulos2015}
                        }\\
                    \hline
                        \specialcell{
                            Spectral Rolloff
                        }&\specialcell{
                            The frequency at which $90\%$ of the spectral magnitude\\
                            distribution lies below: $SR=K$,\\
                            where $\sum_{f=0}^{K} E_K= 0.9\sum_{f=0}^{f_{MAX}}$.
                        }&\specialcell{
                            \code{winlen} $=0.025$\\
                            \code{winstep} $=0.01$\\
                            \code{ratio} $=0.01$: ratio of magnitude \\distribution
                        }&\specialcell{
                            Spectral
                        }&\specialcell{
                            $\mathbb{R}^{1}$
                        }&\specialcell{
                            \cite{Giannakopoulos2015}
                        }\\
                    \hline
                    \hline
                        Terms&\multicolumn{5}{|l|}{\specialcell{
                        In order of usage: 
                        $f$ - frequency in Hertz; 
                        $m(f)$ - magnitude of mel-spectrum at $f$; 
                        $T$ - length of signal (samples); 
                        $t$ - sample index;\\
                        $1_{\mathbb{R}_{<0}}(a)$ - indicator function, 1 if $a<0$, $0$ otherwise; 
                        $s_t$ - value of signal at index $t$; 
                        $E$ - energy; 
                        $H$ - entropy; 
                        sw - subwindow index; \\
                        $S(f)$ = magnitude of Hertz spectrum at $f$; 
                        $w$ - window index; 
                        $K$ - spectral rolloff frequency (Hz).
                        }}\cr
                    \hline
                \end{tabular}
            % }
            \caption{Summary of features implemented in MozzPy.}
            \label{tbl:pl-feats-audio-feattbl}
        \end{table}
        Ten features are selected to be used for this application, giving a feature matrix of dimension $\mathbb{R}^{304}$. Features have been selected based on a number of sources; a similar study with pseudo-acoustic sensors \cite{Silva2013}, previous research in generic species identification \cite{Zilli2016} and the features available in both the `pyAudioAnalysis' \cite{Giannakopoulos2015} and `python\_speech\_features' \cite{Lyons} libraries. Each feature calculated on a window-by window basis, determined by the parameters \code{winlen} and \code{winstep} which dictate the size of the window in seconds and the size of distance between each window in seconds respectively. 
        The chosen features are summarised in table \ref{tbl:pl-feats-audio-feattbl}, where the name, description, type, dimension and source is given for each feature. These have been adapted from existing code to integrate within the MozzPy framework.
        % \begin{sitemize}
        %     \item{list the implemented features with short descriptions}
        %     \item{emphasise important ones eg spectrogram, mfcc, etc}
        %     \item{some maths}
        % \end{sitemize}
        
    \subsection{Software Implementation}
    \label{subsec:pl-feats-software}
        Extracted features are stored in \code{Feature} objects which are in turn stored in a \code{FeatureSet} object. \code{Feature} objects contain individual feature attributes as well as methods to split, merge and sort; whereas a \code{FeatureSet} object provides methods and attributes to aid in management of the entire feature-space. On generation of features, both the features and the associated configuration dictionary are saved to individual files. Subsequent executions will then check the configuration directory before processing and load a previous \code{FeatureSet} if the configurations match. At this stage, the features are not ready for use in classification.  
        % \begin{sitemize}
        %     \item{interface}
        %     \item{modularity}
        %     \item{how features are stored}
        %     % \item{default values, from davs code and that site list}
        % \end{sitemize}
        
        % defaults: https://github.com/jameslyons/python_speech_features
  