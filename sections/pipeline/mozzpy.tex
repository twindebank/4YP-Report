\section{MozzPy}
\label{sec:pl-mozzpy}
    \twN{this section needs finishing, independant of rest of report so can fill in later}
    \subsection{Purpose}
    \label{subsec:pl-mozzpy-purp}
        \begin{sitemize}
            \item{have detailed the steps of the pipeline, this package provides a framework to now carry out research}
            \item{advantages of it are: easy to configure new experiments to run, easy to tweak any details within the pipeline from the config file, etc}
        \end{sitemize}
    
    \subsection{Structure}
    \label{subsec:pl-mozzpy-struc}
        \newcommand{\mozzpyset}[7]{
    % start (int), stop (int), title (str), position (str), colour (str), elipses (bool), unique id (str)
    \node (#7#3set) [blank,#4]{#3Set};
    \node (#7n#1) [small#5,below=0.15cm of #7#3set]{#3\textsubscript{#1}};
    
    \pgfmathsetmacro\diff{int(#2 - #1)}
    \pgfmathsetmacro\startloop{int(#1 + 1)}
    \pgfmathsetmacro\elipses{#6}
    % \node (l#2) [smallblue,below=0.1cm of l#1]{Labels\textsubscript{#2}};
    
    \pgfmathparse{(\diff>0)?1:0}\ifdim\pgfmathresult pt>0pt%
        \foreach \i in {\startloop,...,#2}
            \pgfmathsetmacro\prev{\i - 1}
            \node (#7n\i) [small#5,below left=0.4cm and 0cm of #7n\prev]{#3\textsubscript{\i}};
    \fi
    
    \pgfmathparse{(\elipses>0)?1:0}\ifdim\pgfmathresult pt>0pt%
        \node (#7last) [blank,below=0.1cm of #7n#2]{...}; 
    
        \begin{pgfonlayer}{background}
            \node(#7)[big#5] [fit = (#7n#1) (#7last)] {};
        \end{pgfonlayer}
    \fi
    
    \pgfmathparse{(\elipses==0)?1:0}\ifdim\pgfmathresult pt>0pt%
        \begin{pgfonlayer}{background}
            \node(#7)[big#5] [fit = (#7n#1) (#7n#2)] {};
        \end{pgfonlayer}
    \fi
}

\def\normspacing{0.6cm}
\def\bigspacing{0.8cm} 
\def\largespacing{1.3cm} 

\begin{tikzfig}{fig:pl-mozzpy-struc-dataflow}{Data-flow diagram of MozzPy.}{\tiny}
    \node (audio) [smallazure] {\specialcell{Audio\\Files}};
    \mozzpyset{1}{2}{Wav}{above right=0.5cm and \bigspacing of audio}{darkmagenta}{1}{wavset1}
    \node (labels) [smallblush, below=1.5cm of audio] {\specialcell{Label\\Files}};
    
    % labelset
    \mozzpyset{1}{2}{Label}{above right=0.5cm and \bigspacing of labels}{cadmiumgreen}{1}{labelset1}
    % featureset
    \mozzpyset{1}{2}{Feature}{above right=0cm and \largespacing of wavset1}{darkbyzantium}{1}{featureset1}
    
    % dataset
    \def\setname{ds1}
    \node (\setname-title) [blank,below right=-1.5cm and \largespacing of featureset1]{DataSet};
    \node (\setname-wavset) [smalldarkmagenta, below=0.15cm of \setname-title] {WavSet};
    \node (\setname-featureset) [smalldarkbyzantium, below=0.15cm of \setname-wavset] {FeatureSet};
    \node (\setname-labelset) [smallcadmiumgreen, below=0.15cm of \setname-featureset] {\hspace*{1mm}LabelSet\hspace*{1mm}};
    \begin{pgfonlayer}{background}
        \node(\setname)[bigred] [fit = (\setname-wavset) (\setname-labelset)] {};
    \end{pgfonlayer}
    
    % dataset train
    \node (ds2) [smallred, above right=-0.9cm and \bigspacing of ds1] {\specialcell{DataSet\\Train}};
    
    % dataset test
    \node (ds3) [smallred, below=0.6cm of ds2] {\specialcell{DataSet\\Test}};
    
    % classiset rf
    \def\setname{rf}
    \node (\setname-title) [blank,above right=4.4cm and \normspacing of ds2]{\specialcell{ClassiSet \\ Random Forest}};
    \node (\setname-spacer) [blank,below=0.5mm of \setname-title]{\hspace*{1.2cm}};
    \mozzpyset{2}{2}{Feature}{below=-1.5mm of \setname-spacer}{darkbyzantium}{0}{\setname-featureset}
    \mozzpyset{1}{1}{Label}{below=0.1cm of \setname-featureset}{cadmiumgreen}{0}{\setname-labelset}
    \node (\setname-clf) [smallamber, below=0.15cm of \setname-labelset] {\specialcell{RF\\ Classifier}};
    \node (\setname-output) [smallcaribbeangreen, below=0.15cm of \setname-clf] {Output\textsubscript{1}};
    \node (\setname-results) [smallazure, below=0.15cm of \setname-output] {\specialcell{Test\\Results\textsubscript{1}}};
    \begin{pgfonlayer}{background}
        \node(\setname)[bigblue] [fit = (\setname-spacer) (\setname-results)] {};
    \end{pgfonlayer}
    
   % classiset svm
    \def\setname{svm}
    \node (\setname-title) [blank,below=0.2cm of rf]{\specialcell{ClassiSet \\SVM}};
    \node (\setname-spacer) [blank,below=0.5mm of \setname-title]{\hspace*{1.2cm}};
    \mozzpyset{1}{1}{Feature}{below=-1.5mm of \setname-spacer}{darkbyzantium}{0}{\setname-featureset}
    \mozzpyset{2}{2}{Label}{below=0.1cm of \setname-featureset}{cadmiumgreen}{0}{\setname-labelset}
    \node (\setname-clf) [smallamber, below=0.15cm of \setname-labelset] {\specialcell{SVM \\Classifier}};
    \node (\setname-output) [smallcaribbeangreen, below=0.15cm of \setname-clf] {Output\textsubscript{2}};
    \node (\setname-results) [smallazure, below=0.15cm of \setname-output] {\specialcell{Test\\Results\textsubscript{2}}};
    \begin{pgfonlayer}{background}
        \node(\setname)[bigblue] [fit = (\setname-spacer) (\setname-results)] {};
    \end{pgfonlayer}
    
    % aggset
    \def\setname{as}
    \node (\setname-title) [blank,above right=1.8cm and \largespacing of svm]{\specialcell{AggSet}};
    \node (\setname-output1) [smallcaribbeangreen, below=0.1cm of \setname-title] {Output\textsubscript{1}};
    \node (\setname-output2) [smallcaribbeangreen, below=0.15cm of \setname-output1] {Output\textsubscript{2}};
    \node (\setname-outputagg) [smalldarktangerine, below=0.15cm of \setname-output2] {Output\textsubscript{agg}};
    \node (\setname-results) [smallazure, below=0.15cm of \setname-outputagg] {\specialcell{Test\\Results\textsubscript{agg}}};
    \begin{pgfonlayer}{background}
        \node(\setname)[bigcaribbeangreen] [fit = (\setname-output1) (\setname-results)] {};
    \end{pgfonlayer}
    
    % resultset
    \def\setname{rs}
    \node (\setname-title) [blank,above right=-2mm and \largespacing of as]{ResultSet};
    \node (\setname-results1) [smallazure, below=0.15cm of \setname-title] {\specialcell{Test\\Results\textsubscript{1}}};
    \node (\setname-results2) [smallazure, below=0.15cm of \setname-results1] {\specialcell{Test\\Results\textsubscript{2}}};
    \node (\setname-resultsagg) [smallazure, below=0.15cm of \setname-results2] {\specialcell{Test\\Results\textsubscript{agg}}};
    \begin{pgfonlayer}{background}
        \node(\setname)[bigazure] [fit = (\setname-results1) (\setname-resultsagg)] {};
    \end{pgfonlayer}
    
    \node(plot)[blank, below=0.6cm of rs]{plot};
    
    
    % arrows
    \draw [arrow] (audio) -- (wavset1) node[midway,above]{import};
    \draw [arrow] (labels) -- (labelset1) node[midway,above]{import};
    \draw [arrow] (wavset1) -- (featureset1) node[midway,below]{\specialcell{generate \\all features}};
    \draw [arrow] (wavset1.north) |- (2,2) -| (ds1-wavset.north);
    \draw [arrow] (featureset1.east) -| (5,-0.5) |- (ds1-featureset.west);
    \draw [arrow] (labelset1.east) -| (5,-2) |- (ds1-labelset.west);
    \draw [arrow] (ds1.east) -|  node[midway,above]{split\hspace*{0.6cm}} (7.7,-0.5) |- (ds2.west);
    \draw [arrow] (ds1.east) -|  (7.7,-1.5) |- (ds3.west);
    \draw [arrow] (ds2.north) |-  node[near end,above]{\hspace{-1cm}pre-process features} (rf-featureset.west);
    \draw [arrow] (ds2.north) |-  node[near end,above]{\hspace{-3cm}resample lables} (rf-labelset.west);
    \draw [arrow] (rf-featureset.east) -|  (12,2.3) node[near start,above]{\hspace*{1cm} train classifier} |- (rf-clf.east);
    \draw [arrow] (rf-labelset.east) -|  (12,2.3)  |- (rf-clf.east);
    \draw [arrow] (rf-clf) -- (rf-output);
    \draw [arrow] (rf-output) -- (rf-results);
    \draw[arrow] (rf-output.east) -| (11.7,0.5) |- (as-output1);
    \draw [arrow] (ds3.east) -| (9,1) |- node[midway,above]{\hspace{1mm} test classifier}  (rf-clf);
     \draw [arrow] (ds3.east) -| (9,-2) |- (svm-clf);
    \draw [arrow] (ds2.east) -| (9.3,-2) |- (svm-labelset.west);
    \draw [arrow] (ds2.east) -| (9.3,-2) |- (svm-featureset.west);
    \draw[arrow] (svm-featureset.east) -| (11.4,-3) |- (svm-clf.east);
    \draw[arrow] (svm-labelset.east) -| (11.4,-3.5) |- (svm-clf.east);
    \draw[arrow] (svm-clf) -- (svm-output);
    \draw[arrow] (svm-output) -- (svm-results);
    \draw[arrow] (svm-output.east) -| (11.7,-3) |- (as-output2);
    \draw[arrow] (as-output1.east) -| (14,-0.5) |- (as-outputagg.east);
    \draw[arrow] (as-output2.east) -| (14,-0.5) |- (as-outputagg.east);
    \draw[arrow] (as-outputagg) -- (as-results);
    \draw[arrow] (rf-results.east) -| (11.4,1) |- (16.4,1) |- (rs-results1);
    \draw[arrow] (svm-results.east) -| (16.4,-2) |- (rs-results2.east);
    \draw[arrow] (as-results) -| (14,-2.5) |- (rs-resultsagg);
    \draw[arrow] (rs) -- (plot);
\end{tikzfig}

        \twN{need to fix diagram, slight misalignment}
        Figure \ref{fig:pl-mozzpy-struc-dataflow} depicts how data is passed through the MozzPy package. Two classification pipelines are shown in this simplified example, in which they both use different sets of features and labels combined using different methods described in section \ref{subsec:pl-data-software}. The diagram is simplified for the purpose of clarity, the composition/inheritance relationship is shown in figure \ref{fig:pl-mozzpy-struc-inhercomp}.
        
        \begin{tikzfig}{fig:pl-mozzpy-struc-inhercomp}{Composition and inheritance of MozzPy.}{\tiny}
    \node (humset) [smallazure] {HumSet};
    \node (featureset) [smallcadmiumgreen, below=0.5cm of humset] {FeatureSet};
    \node (wavset) [smallcadmiumgreen, left=0.5cm of featureset] {WavSet};
    \node (labelset) [smallcadmiumgreen, right=0.5cm of featureset] {LabelSet};
    \node (dataset) [smallcadmiumgreen, below=0.5cm of featureset] {DataSet};
    \node (predset) [smallazure, right=0.5cm of labelset] {PredSet};
    \node (classiset) [smallcadmiumgreen, below left=0.5cm and -0.3cm of predset] {ClassiSet};
    \node (aggset) [smallcadmiumgreen, right=0.5cm of classiset] {AggSet};
    \node (resultset) [smallcadmiumgreen, below left=0.5cm and -0.3cm of aggset] {ResultSet};
    % arrows
    \draw [orangearrow] (humset.south) |- (0,-0.5) -| (wavset.north);
    \draw [orangearrow] (humset.south) |- (0,-0.5) -| (featureset.north);
    \draw [orangearrow] (humset.south) |- (0,-0.5) -| (labelset.north);
    \draw [orangearrow] (humset.south) |- (0,-0.5) -| (predset.north);
    \draw [orangearrow] (humset.south) |- (-2.5,-0.5) -| (-2.5,-1) |- (dataset.west);
    \draw [orangearrow] (humset.south) |- (5,-0.5) -| (5,-2) |- (resultset.east);
    
    \draw [purplearrow] (featureset) -- (dataset);
    \draw [purplearrow] (wavset.south) |- (0,-1.6) -| (dataset.north);
    \draw [purplearrow] (labelset.south) |- (0,-1.6) -| (dataset.north);
    
    \draw [orangearrow] (predset.south) |- (3.5,-1.6) -| (classiset.north);
    \draw [orangearrow] (predset.south) |- (3.5,-1.6) -| (aggset.north);
    
    \draw [purplearrow] (classiset) -- (aggset);
    \draw [purplearrow] (dataset) -- (classiset);
    
    \draw [purplearrow] (classiset.south) |- (3.5,-2.7) -| (resultset.north);
    \draw [purplearrow] (aggset.south) |- (3.5,-2.7) -| (resultset.north);
    
    % key
    \node (keytitle) [blank,left=5cm of humset] {Key};
    \node (key1a) [blankdot, below left=2mm and -1.5mm of keytitle] {};
    \node (key1b) [blankdot, right=0.3cm of key1a] {~~Composition (has-a)};
    \draw [purplearrow] (key1a) -- (key1b);
    
    \node (key2a) [blankdot, below=0.2cm of key1a] {};
    \node (key2b) [blankdot, right=0.3cm of key2a] {~~Inheritance (is-a)};
    \draw [orangearrow] (key2a) -- (key2b);
    
    \node (key3) [smallazure, below right=0.2cm and 0cm of key2a, scale=0.8] {Abstract Base Class};

    \node (key4) [smallcadmiumgreen, below right=0.8cm and 0cm of key2a, scale=0.8] { Regular Class};
    \begin{pgfonlayer}{background}
        \node(key)[bigcaribbeangreen] [fit = (key1b) (key4)] {};
    \end{pgfonlayer}
    
    % signal
    \node (signal) [smallcadmiumgreen, right=7cm of humset] {Signal};
    \node (wav) [smallcadmiumgreen, below=0.5cm of signal] {Wav};
    \node (label) [smallcadmiumgreen, left=0.5cm of wav] {Label};
    \node (prediction) [smallcadmiumgreen, right=0.5cm of wav] {Prediction};
    \draw [orangearrow] (signal.south) |- (8,-0.5) -| (wav.north);
    \draw [orangearrow] (signal.south) |- (8,-0.5) -| (prediction.north);
    \draw [orangearrow] (signal.south) |- (8,-0.5) -| (label.north);
    
\end{tikzfig}
        
  
        
    
    \subsection{Usage}
    \label{subsec:pl-mozzpy-usage}
        \begin{sitemize}
            \item{show code snippet of how to run a exp and explain how will be detailed in this section}
            \item{explain how to configure a pipeline and how simple it is to run multiple experiments}
            \item{saving and laoding with file locking}
        \end{sitemize}
   
    \subsection{Limitations}
    \label{subsec:pl-mozzpy-limit}
        \begin{sitemize}
            \item{current limitations: written in 2.7 not 3.5 - can be ported, wrote tailored to mozz problem so limited for use in other contexts to binary time series classification, functionality could be extended with some refactoring}
        \end{sitemize}

