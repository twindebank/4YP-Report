\section{MozzPy}
\label{sec:pl-mozzpy}
    \subsection{Purpose}
    \label{subsec:pl-mozzpy-purp}
        Given the pipeline established in figure \ref{fig:pl-overview-general-diag}, experiments must be designed and executed to gather results. Many scripts could be written to achieve this; however, to fully conform to the established pipeline and significantly speed up the process, `MozzPy' is built. This package provides a modular framework in which code, conforming to a specified interface, can be added at any stage of the pipeline leading to the easy design and running of new experiments.
    
    \subsection{Structure}
    \label{subsec:pl-mozzpy-struc}
        \newcommand{\mozzpyset}[7]{
    % start (int), stop (int), title (str), position (str), colour (str), elipses (bool), unique id (str)
    \node (#7#3set) [blank,#4]{#3Set};
    \node (#7n#1) [small#5,below=0.15cm of #7#3set]{#3\textsubscript{#1}};
    
    \pgfmathsetmacro\diff{int(#2 - #1)}
    \pgfmathsetmacro\startloop{int(#1 + 1)}
    \pgfmathsetmacro\elipses{#6}
    % \node (l#2) [smallblue,below=0.1cm of l#1]{Labels\textsubscript{#2}};
    
    \pgfmathparse{(\diff>0)?1:0}\ifdim\pgfmathresult pt>0pt%
        \foreach \i in {\startloop,...,#2}
            \pgfmathsetmacro\prev{\i - 1}
            \node (#7n\i) [small#5,below left=0.4cm and 0cm of #7n\prev]{#3\textsubscript{\i}};
    \fi
    
    \pgfmathparse{(\elipses>0)?1:0}\ifdim\pgfmathresult pt>0pt%
        \node (#7last) [blank,below=0.1cm of #7n#2]{...}; 
    
        \begin{pgfonlayer}{background}
            \node(#7)[big#5] [fit = (#7n#1) (#7last)] {};
        \end{pgfonlayer}
    \fi
    
    \pgfmathparse{(\elipses==0)?1:0}\ifdim\pgfmathresult pt>0pt%
        \begin{pgfonlayer}{background}
            \node(#7)[big#5] [fit = (#7n#1) (#7n#2)] {};
        \end{pgfonlayer}
    \fi
}

\def\normspacing{0.6cm}
\def\bigspacing{0.8cm} 
\def\largespacing{1.3cm} 

\begin{tikzfig}{fig:pl-mozzpy-struc-dataflow}{Data-flow diagram of MozzPy.}{\tiny}
    \node (audio) [smallazure] {\specialcell{Audio\\Files}};
    \mozzpyset{1}{2}{Wav}{above right=0.5cm and \bigspacing of audio}{darkmagenta}{1}{wavset1}
    \node (labels) [smallblush, below=1.5cm of audio] {\specialcell{Label\\Files}};
    
    % labelset
    \mozzpyset{1}{2}{Label}{above right=0.5cm and \bigspacing of labels}{cadmiumgreen}{1}{labelset1}
    % featureset
    \mozzpyset{1}{2}{Feature}{above right=0cm and \largespacing of wavset1}{darkbyzantium}{1}{featureset1}
    
    % dataset
    \def\setname{ds1}
    \node (\setname-title) [blank,below right=-1.5cm and \largespacing of featureset1]{DataSet};
    \node (\setname-wavset) [smalldarkmagenta, below=0.15cm of \setname-title] {WavSet};
    \node (\setname-featureset) [smalldarkbyzantium, below=0.15cm of \setname-wavset] {FeatureSet};
    \node (\setname-labelset) [smallcadmiumgreen, below=0.15cm of \setname-featureset] {\hspace*{1mm}LabelSet\hspace*{1mm}};
    \begin{pgfonlayer}{background}
        \node(\setname)[bigred] [fit = (\setname-wavset) (\setname-labelset)] {};
    \end{pgfonlayer}
    
    % dataset train
    \node (ds2) [smallred, above right=-0.9cm and \bigspacing of ds1] {\specialcell{DataSet\\Train}};
    
    % dataset test
    \node (ds3) [smallred, below=0.6cm of ds2] {\specialcell{DataSet\\Test}};
    
    % classiset rf
    \def\setname{rf}
    \node (\setname-title) [blank,above right=4.4cm and \normspacing of ds2]{\specialcell{ClassiSet \\ Random Forest}};
    \node (\setname-spacer) [blank,below=0.5mm of \setname-title]{\hspace*{1.2cm}};
    \mozzpyset{2}{2}{Feature}{below=-1.5mm of \setname-spacer}{darkbyzantium}{0}{\setname-featureset}
    \mozzpyset{1}{1}{Label}{below=0.1cm of \setname-featureset}{cadmiumgreen}{0}{\setname-labelset}
    \node (\setname-clf) [smallamber, below=0.15cm of \setname-labelset] {\specialcell{RF\\ Classifier}};
    \node (\setname-output) [smallcaribbeangreen, below=0.15cm of \setname-clf] {Output\textsubscript{1}};
    \node (\setname-results) [smallazure, below=0.15cm of \setname-output] {\specialcell{Test\\Results\textsubscript{1}}};
    \begin{pgfonlayer}{background}
        \node(\setname)[bigblue] [fit = (\setname-spacer) (\setname-results)] {};
    \end{pgfonlayer}
    
   % classiset svm
    \def\setname{svm}
    \node (\setname-title) [blank,below=0.2cm of rf]{\specialcell{ClassiSet \\SVM}};
    \node (\setname-spacer) [blank,below=0.5mm of \setname-title]{\hspace*{1.2cm}};
    \mozzpyset{1}{1}{Feature}{below=-1.5mm of \setname-spacer}{darkbyzantium}{0}{\setname-featureset}
    \mozzpyset{2}{2}{Label}{below=0.1cm of \setname-featureset}{cadmiumgreen}{0}{\setname-labelset}
    \node (\setname-clf) [smallamber, below=0.15cm of \setname-labelset] {\specialcell{SVM \\Classifier}};
    \node (\setname-output) [smallcaribbeangreen, below=0.15cm of \setname-clf] {Output\textsubscript{2}};
    \node (\setname-results) [smallazure, below=0.15cm of \setname-output] {\specialcell{Test\\Results\textsubscript{2}}};
    \begin{pgfonlayer}{background}
        \node(\setname)[bigblue] [fit = (\setname-spacer) (\setname-results)] {};
    \end{pgfonlayer}
    
    % aggset
    \def\setname{as}
    \node (\setname-title) [blank,above right=1.8cm and \largespacing of svm]{\specialcell{AggSet}};
    \node (\setname-output1) [smallcaribbeangreen, below=0.1cm of \setname-title] {Output\textsubscript{1}};
    \node (\setname-output2) [smallcaribbeangreen, below=0.15cm of \setname-output1] {Output\textsubscript{2}};
    \node (\setname-outputagg) [smalldarktangerine, below=0.15cm of \setname-output2] {Output\textsubscript{agg}};
    \node (\setname-results) [smallazure, below=0.15cm of \setname-outputagg] {\specialcell{Test\\Results\textsubscript{agg}}};
    \begin{pgfonlayer}{background}
        \node(\setname)[bigcaribbeangreen] [fit = (\setname-output1) (\setname-results)] {};
    \end{pgfonlayer}
    
    % resultset
    \def\setname{rs}
    \node (\setname-title) [blank,above right=-2mm and \largespacing of as]{ResultSet};
    \node (\setname-results1) [smallazure, below=0.15cm of \setname-title] {\specialcell{Test\\Results\textsubscript{1}}};
    \node (\setname-results2) [smallazure, below=0.15cm of \setname-results1] {\specialcell{Test\\Results\textsubscript{2}}};
    \node (\setname-resultsagg) [smallazure, below=0.15cm of \setname-results2] {\specialcell{Test\\Results\textsubscript{agg}}};
    \begin{pgfonlayer}{background}
        \node(\setname)[bigazure] [fit = (\setname-results1) (\setname-resultsagg)] {};
    \end{pgfonlayer}
    
    \node(plot)[blank, below=0.6cm of rs]{plot};
    
    
    % arrows
    \draw [arrow] (audio) -- (wavset1) node[midway,above]{import};
    \draw [arrow] (labels) -- (labelset1) node[midway,above]{import};
    \draw [arrow] (wavset1) -- (featureset1) node[midway,below]{\specialcell{generate \\all features}};
    \draw [arrow] (wavset1.north) |- (2,2) -| (ds1-wavset.north);
    \draw [arrow] (featureset1.east) -| (5,-0.5) |- (ds1-featureset.west);
    \draw [arrow] (labelset1.east) -| (5,-2) |- (ds1-labelset.west);
    \draw [arrow] (ds1.east) -|  node[midway,above]{split\hspace*{0.6cm}} (7.7,-0.5) |- (ds2.west);
    \draw [arrow] (ds1.east) -|  (7.7,-1.5) |- (ds3.west);
    \draw [arrow] (ds2.north) |-  node[near end,above]{\hspace{-1cm}pre-process features} (rf-featureset.west);
    \draw [arrow] (ds2.north) |-  node[near end,above]{\hspace{-3cm}resample lables} (rf-labelset.west);
    \draw [arrow] (rf-featureset.east) -|  (12,2.3) node[near start,above]{\hspace*{1cm} train classifier} |- (rf-clf.east);
    \draw [arrow] (rf-labelset.east) -|  (12,2.3)  |- (rf-clf.east);
    \draw [arrow] (rf-clf) -- (rf-output);
    \draw [arrow] (rf-output) -- (rf-results);
    \draw[arrow] (rf-output.east) -| (11.7,0.5) |- (as-output1);
    \draw [arrow] (ds3.east) -| (9,1) |- node[midway,above]{\hspace{1mm} test classifier}  (rf-clf);
     \draw [arrow] (ds3.east) -| (9,-2) |- (svm-clf);
    \draw [arrow] (ds2.east) -| (9.3,-2) |- (svm-labelset.west);
    \draw [arrow] (ds2.east) -| (9.3,-2) |- (svm-featureset.west);
    \draw[arrow] (svm-featureset.east) -| (11.4,-3) |- (svm-clf.east);
    \draw[arrow] (svm-labelset.east) -| (11.4,-3.5) |- (svm-clf.east);
    \draw[arrow] (svm-clf) -- (svm-output);
    \draw[arrow] (svm-output) -- (svm-results);
    \draw[arrow] (svm-output.east) -| (11.7,-3) |- (as-output2);
    \draw[arrow] (as-output1.east) -| (14,-0.5) |- (as-outputagg.east);
    \draw[arrow] (as-output2.east) -| (14,-0.5) |- (as-outputagg.east);
    \draw[arrow] (as-outputagg) -- (as-results);
    \draw[arrow] (rf-results.east) -| (11.4,1) |- (16.4,1) |- (rs-results1);
    \draw[arrow] (svm-results.east) -| (16.4,-2) |- (rs-results2.east);
    \draw[arrow] (as-results) -| (14,-2.5) |- (rs-resultsagg);
    \draw[arrow] (rs) -- (plot);
\end{tikzfig}

        \begin{tikzfig}{fig:pl-mozzpy-struc-inhercomp}{Composition and inheritance of MozzPy.}{\tiny}
    \node (humset) [smallazure] {HumSet};
    \node (featureset) [smallcadmiumgreen, below=0.5cm of humset] {FeatureSet};
    \node (wavset) [smallcadmiumgreen, left=0.5cm of featureset] {WavSet};
    \node (labelset) [smallcadmiumgreen, right=0.5cm of featureset] {LabelSet};
    \node (dataset) [smallcadmiumgreen, below=0.5cm of featureset] {DataSet};
    \node (predset) [smallazure, right=0.5cm of labelset] {PredSet};
    \node (classiset) [smallcadmiumgreen, below left=0.5cm and -0.3cm of predset] {ClassiSet};
    \node (aggset) [smallcadmiumgreen, right=0.5cm of classiset] {AggSet};
    \node (resultset) [smallcadmiumgreen, below left=0.5cm and -0.3cm of aggset] {ResultSet};
    % arrows
    \draw [orangearrow] (humset.south) |- (0,-0.5) -| (wavset.north);
    \draw [orangearrow] (humset.south) |- (0,-0.5) -| (featureset.north);
    \draw [orangearrow] (humset.south) |- (0,-0.5) -| (labelset.north);
    \draw [orangearrow] (humset.south) |- (0,-0.5) -| (predset.north);
    \draw [orangearrow] (humset.south) |- (-2.5,-0.5) -| (-2.5,-1) |- (dataset.west);
    \draw [orangearrow] (humset.south) |- (5,-0.5) -| (5,-2) |- (resultset.east);
    
    \draw [purplearrow] (featureset) -- (dataset);
    \draw [purplearrow] (wavset.south) |- (0,-1.6) -| (dataset.north);
    \draw [purplearrow] (labelset.south) |- (0,-1.6) -| (dataset.north);
    
    \draw [orangearrow] (predset.south) |- (3.5,-1.6) -| (classiset.north);
    \draw [orangearrow] (predset.south) |- (3.5,-1.6) -| (aggset.north);
    
    \draw [purplearrow] (classiset) -- (aggset);
    \draw [purplearrow] (dataset) -- (classiset);
    
    \draw [purplearrow] (classiset.south) |- (3.5,-2.7) -| (resultset.north);
    \draw [purplearrow] (aggset.south) |- (3.5,-2.7) -| (resultset.north);
    
    % key
    \node (keytitle) [blank,left=5cm of humset] {Key};
    \node (key1a) [blankdot, below left=2mm and -1.5mm of keytitle] {};
    \node (key1b) [blankdot, right=0.3cm of key1a] {~~Composition (has-a)};
    \draw [purplearrow] (key1a) -- (key1b);
    
    \node (key2a) [blankdot, below=0.2cm of key1a] {};
    \node (key2b) [blankdot, right=0.3cm of key2a] {~~Inheritance (is-a)};
    \draw [orangearrow] (key2a) -- (key2b);
    
    \node (key3) [smallazure, below right=0.2cm and 0cm of key2a, scale=0.8] {Abstract Base Class};

    \node (key4) [smallcadmiumgreen, below right=0.8cm and 0cm of key2a, scale=0.8] { Regular Class};
    \begin{pgfonlayer}{background}
        \node(key)[bigcaribbeangreen] [fit = (key1b) (key4)] {};
    \end{pgfonlayer}
    
    % signal
    \node (signal) [smallcadmiumgreen, right=7cm of humset] {Signal};
    \node (wav) [smallcadmiumgreen, below=0.5cm of signal] {Wav};
    \node (label) [smallcadmiumgreen, left=0.5cm of wav] {Label};
    \node (prediction) [smallcadmiumgreen, right=0.5cm of wav] {Prediction};
    \draw [orangearrow] (signal.south) |- (8,-0.5) -| (wav.north);
    \draw [orangearrow] (signal.south) |- (8,-0.5) -| (prediction.north);
    \draw [orangearrow] (signal.south) |- (8,-0.5) -| (label.north);
    
\end{tikzfig}
        Figure \ref{fig:pl-mozzpy-struc-dataflow} depicts how data is passed through the MozzPy package. Two classification pipelines are shown in this simplified example, in which they both use different sets of features and labels combined using different methods, explained further in section \ref{subsec:pl-data-software}. The diagram is simplified for the purpose of clarity, the composition/inheritance relationship is shown in figure \ref{fig:pl-mozzpy-struc-inhercomp}. This section will follow the flow of the data-flow diagram, detailing both theory and how the data is handled through each stage.
  
        
    
    \subsection{Usage}
    \label{subsec:pl-mozzpy-usage}
        Usage of the code-base is based around the user following conventions, although this is not a necessity for it to function. Code is separated into two files, an execution script and a configuration file. The configuration file contains nested dictionaries within a \code{Settings} instance; this class defines the data members that hold preferences such as file directories, classifiers to use, tests to run and graphs to plot as well as methods to update values and copy a configuration. When passing configurations to modules it is only necessary to specify arguments that stray from the default values which are specified in the modules, saving time and increasing readability. Settings files are imported into an execution script, an example of which is shown in listing \ref{code:pl-mozzpy-usage-exec}.

        \begin{listing}[ht]
            \begin{minted}[frame=single,framesep=2mm, bgcolor=black!2, fontsize=\footnotesize, linenos]{python}
from MozzPy.utils import WavSet, FeatureSet, ClassiSet, AggSet, ResultSet   
from MozzPy.scripts.conf.example import conf        # import configuration file
wavs = WavSet.fromfiles(**conf.wavs)                # import wav files
wavs.preproc(**conf.wav_preproc)                    # process wav files (resample, normalise, etc)
lbls = LabelSet.fromfiles(wavs,**conf.labels)       # import label files and organise wrt wav files
feats = FeatureSet.fromwavset(wavs,**conf.features) # generate features from wav files
dataset = DataSet(wavs,feats,lbls)                  # collect the data into one object
ds_trn,ds_tst = dataset.discretesplit()         # split randomly, keeping samples grouped by wav
classisets = []                                 # list to fill with ClassiSet objects
for clf_id in conf.clf_init.keys():             # for each classifier config defined in config file
    cs = ClassiSet(**conf.clf_init[clf_id])     # initialise classifier object eg RF with 100 trees
    cs.train(ds_trn,**conf.clf_trn[clf_id])     # train on half the data
    cs.apply(ds_tst,**conf.clf_apply[clf_id])   # generate predictions from the other half
    cs.postproc(**conf.res_postproc[clf_id])    # post-process predictions
    cs.test(**conf.clf_tst[clf_id])             # apply tests to any predictions eg f1 score, ROC
    classisets.append(cs)                       # append to list
aggsets = []                                    # list to fill with aggregated results
for agg_id in conf.aggregate.keys():            # for each aggregation method defined in config
    aggset = AggSet(agg_id, conf.aggregate[agg_id], *classisets)    # aggregate results
    aggset.postproc(**conf.agg_postproc[agg_id])                    # post-process agg`d results
    aggset.test(**conf.agg_tst[agg_id])         # test aggregated results
    aggsets.append(aggset)                      # append to list
predsets = classisets + aggsets                 # gather all predictions together
rs = ResultSet(*predsets)                       # put into ResultSet object
rs.plot(**conf.resultset_plot)                  # plot graphs specified in config
rs.json_exprt(**conf.json)                      # export info to json file
rs.mongo_exprt(**conf.mongo)                    # export info to mongo database
            \end{minted}
            \caption{Example MozzPy execution script.}
            \label{code:pl-mozzpy-usage-exec}
        \end{listing} 
    
        During both feature generation and classifier training, the program first checks to see if the same configuration has been run previously. If it has, then the previous data is loaded, reducing runtimes for repeated experiments dramatically. Parallel implementations are safe during saving/loading as the directory is locked when it is being read or written to.
   
    % \subsection{Limitations}
    % \label{subsec:pl-mozzpy-limit}
    %     MozzPy is written in Python 2.7, this is considered a limitation as many shortcomings of 2.7 have been fixed in the more current Python 3.5. The older version was chosen to ease integration as existing code within the HumBug project using Python 2.5. The package is coded completely tailored to the problem of mosquito classification with .wav files and labels. This is an unnecessary limitation as there may be potential uses of this utility outside of the mosquito-detection domain. These limitations are not overly inhibiting and are able to be solved with some degree of refactoring.
