\section{Testing}
\label{sec:pl-test}
    The desired output is restricted to the binary domain, restricting/defining how analysis metrics will be calculated.
    \subsection{Standard Machine Learning Metrics}
    \label{subsec:pl-test-stan}
        \begin{sitemize}
            \item{confusion mat, roc, prec-rec, f1 score [harmonic mean, makes more sense for ratios and rates], crossval}
            %https://www.quora.com/What-is-an-intuitive-explanation-of-F-score
            \item{explain what each describes and its relevancy in this context}
            \item{explain significance of areas as well and how they can give good condensed idea of graph}
            \item{test of RFE and PCA, explain how only some classifiers can do rfe, talk about log and linear spacing}
            \item{', but also the area under the ROC curve (AUC), more recommended for evaluating the classification in problems involving imbalanced classes [22].'}
        \end{sitemize}
        
    \subsection{Less-Common Machine Learning Metrics}
    \label{subsec:pl-test-less}
        \begin{sitemize}
            \item{rejection ratio, acc-rej plot, f1-rej plot, binary asymmetrical rejection matrix,median-f1,etc}
            \item{explain what each describes and its relevancy in this context}
        \end{sitemize}
    
    \subsection{Software Implementation}
    \label{subsec:pl-test-software}
        \begin{sitemize}
            \item{testing only on binary labels as that is how problem is formed}
            \item{plotting
            These objects are stored in a \code{WavSet} object which implements a \code{plot()} method which calls the abstract meth\code{subplot()}, talk about the polymorphism here}
            \item{rejection plot overlay, gaussian normalised spikes}
            \item{presenting results}
        \end{sitemize}