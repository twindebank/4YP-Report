\section{Introduction}
\label{sec:bg-intro}

    \subsection{Research Interests}
    \label{subsec:bg-intro-resinterests}
        A living organism that can transmit disease is known as a vector. Vectors, when feeding on blood, ingest disease-producing microorganisms from an infected host. The disease is then later introduced into a new host, human or animal, during subsequent feeding. 
        Mosquitoes are the most well-known disease vector, known to be capable of carrying many diseases depending on the species \cite{WHOVectorBorneDisease2016}.
        Annually, up to 700 million people are infected by mosquito-borne diseases, more than a million of which cases are fatal \cite{Caraballo2014}.
        
        In an effort to reduce these numbers, it is imperative that focus is placed on prevention of disease contraction as much as it is on treatment of existing and future cases. Many methods of prevention exist, divisible into two categories: vector control and personal protective measures \cite{Caraballo2014}. A tool (or network of tools) for mosquito detection, within a localised region, would augment existing preventative measures and likely increase effectiveness. Three areas in which such a detector would positively influence control are:
        \begin{itemize}
            \item{\textbf{Solar Lamp}, \textit{personal protective measure} - Solar house systems are being used increasingly in developing countries to provide basic needs such as lighting and water sterilisation \cite{Akikur2013}. Evidence suggests that an increase in artificial light leads to an increase in malaria through both acting as a vector attractant and changing the human population behaviour \cite{Barghini2010}. A detection device, inbuilt into the solar lamp, would act as a warning system of a dangerous vector presence, allowing the user to take appropriate action.}
            \item{\textbf{Swarm Tracking}, \textit{vector control} - Larval control (source reduction), targeted residual spraying and habitat reduction require knowledge of mosquito location \cite{Caraballo2014,Pates2005,WHOVectorControl2016}. A networked array of sensing/detection devices would aid in localising swarms and increase effectiveness of these control methods. Tracking data could also be used in unison with satellite imagery to provide valuable insights into vector species ecology, this is a goal of the HumBug project \cite{HumBug2016}.}
            \item{\textbf{Data Logging}, \textit{both} - mosquito detection algorithms and general research on the subject require lots of data, a weaker algorithm could be used on a device in unison with a light trap to trigger sensors to capture more data.}
            \item{\textbf{Smartphone}, \textit{personal protective measure} - having an algorithm able to detect mosquito presence using an app would be a potential use for people with access to a smartphone, which carry a useful array of in-built sensors. Aside from residents of affected areas, this would apply to tourists, especially at a large event such as the 2016 Olympics hosted in Rio where Zika was a potential threat.}
        \end{itemize}
        Implications of these use cases on the detector design will be discussed in section \ref{sec:exp-spec}. Based on these cases, it is clear there is a need for both a hardware solution and accompanying algorithms, capable of detecting mosquitoes within the local vicinity. This report will focus on algorithm design, results of which improve with the amount of data available. Microphones, being inexpensive and versatile, provide a suitable method for capturing mosquito characteristics that are usable for detection and identification purposes, discussed further in sec X. Many difficulties arise in the detection and classification of mosquitoes using audio, the most prominent being that mosquitoes are naturally quiet insects, the sounds of which vary with many intrinsic and environmental factors.
        
        Other data-capture methods have been used in this context, based on pseudo-acoustic optical sensors \cite{Chen2014}. Higher signal-to-noise ratios (SNRs) are attained using sensors of this type, at the expense of a complex experimental setup requiring mosquitoes to directly intercept optical beams, drastically limiting the versatility of the detector as well as limiting recordings to snippets of 100ms, with no capability to capture continuous flight. 
        
        % In 2015, malaria accounted for $\sim$212 million of these mosquito-borne infections, and lead to $\sim$429,000 deaths.
        
        % % In partnership with Kew Gardens, the HumBug project has procured a database of over
        
        % Caraballo2014 - mos borne illness, stats on global infection and stats on malaria, dengue, west nile, big focus on travelling and disease spread
        
        % WHO2016 - vector bone disease stats and info
  
    \subsection{Report Structure}
    \label{subsec:bg-intro-structure}
        % The report is organised into a structure designed to present the findings of this report in a 
        The report is divided into three sections. Section \textbf{II} details each stage of the general mosquito detection pipeline, first going into the underlying theory then discussing details of software implementation. Section \textbf{III} describes how performance is improved from base results by optimisation at each stage of the general pipeline. Optimising locally at each stage makes this a \textit{'greedy'} pipeline i.e. it doesn't necessarily converge to the global minimum where performance is best. Finding the global minimum would require a grid-search over the entire parameter space across the whole pipeline. Using a random forest only, \textit{without any post-processing of results considered}, this would require training and testing the random forest tens of millions of times, requiring thousands of days of compute time if done in serial. 
        % random forest decisions [normalise [4], labels [6], features[47], pca/rfe [100], normalise[4], ntrees [20], backend [2], ]
        For this problem, the locally optimised pipeline should be sufficient. Section \textbf{IV} presents the best performing configuration for each use case. Finally, section \textbf{V} concludes the the report and highlights further potential of this research as well as uses for other applications. 
