\section{Introduction}
\label{sec:bg-intro}

    \subsection{Research Interests}
    \label{subsec:bg-intro-resinterests}
        A living organism that can transmit disease is known as a vector. Vectors, when feeding on blood, ingest disease-producing microorganisms from an infected host. The disease is then later introduced into a new host, human or animal, during subsequent feeding. 
        Mosquitoes are the most well-known disease vector, known to be capable of carrying many diseases depending on the species \cite{WHOVectorBorneDisease2016}.
        Annually, up to 700 million people are infected by mosquito-borne diseases, more than a million of which cases are fatal \cite{Caraballo2014}.
        
        In an effort to reduce these numbers, it is imperative that focus is placed on prevention of disease contraction as much as it is on treatment of existing and future cases. Many methods of prevention exist, divisible into two categories: vector control and personal protective measures \cite{Caraballo2014}. A tool (or network of tools) for mosquito detection, within a localised region, would augment existing preventative measures and likely increase effectiveness. Three areas in which such a detector would positively influence control are:
        \begin{itemize}
            \item{\textbf{Solar Lamp}, \textit{personal protective measure} - Solar house systems are being used increasingly in developing countries to provide basic needs such as lighting and water sterilisation \cite{Akikur2013}. Evidence suggests that an increase in artificial light leads to an increase in malaria through both acting as a vector attractant and changing the human population behaviour \cite{Barghini2010}. A detection device, inbuilt into the solar lamp, would act as a warning system of a dangerous vector presence, allowing the user to take appropriate action.}
            \item{\textbf{Swarm Tracking}, \textit{vector control} - Larval control (source reduction), targeted residual spraying and habitat reduction require knowledge of mosquito location \cite{Caraballo2014,Pates2005,WHOVectorControl2016}. A networked array of sensing/detection devices would aid in localising swarms and increase effectiveness of these control methods. Tracking data could also be used in unison with satellite imagery to provide valuable insights into vector species ecology, this is a goal of the HumBug project \cite{HumBug2016}.}
            \item{\textbf{Data Logging}, \textit{both} - audio detection algorithms and general research on the subject require lots of data, a weaker algorithm could be used on a device in unison with a light trap to trigger a microphone in order to capture more data.}
            \item{\textbf{Smartphone}, \textit{personal protective measure} - having an algorithm able to detect mosquito presence using an app would be a potential use for people with access to a smartphone. Aside from residents of affected areas, this would apply to tourists, especially at a large event such as the 2016 Olympics hosted in Rio where Zika was a potential threat.}
        \end{itemize}
        Implications of these use cases on the detector design will be discussed in section \ref{sec:exp-spec}. Based on these cases, it is clear there is a need for a device and algorithms capable of detecting mosquitoes within its local vicinity. This report will focus on the algorithm design.
        
        % In 2015, malaria accounted for $\sim$212 million of these mosquito-borne infections, and lead to $\sim$429,000 deaths.
        
        % % In partnership with Kew Gardens, the HumBug project has procured a database of over
        
        % Caraballo2014 - mos borne illness, stats on global infection and stats on malaria, dengue, west nile, big focus on travelling and disease spread
        
        % WHO2016 - vector bone disease stats and info
  
    \subsection{Report Structure}
    \label{subsec:bg-intro-structure}
        \begin{itemize}
            \item{first detail whole pipeline and software}
            \item{then detail results, variations of pipelines, etc and final good results}
        \end{itemize}
