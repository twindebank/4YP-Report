\section*{Abstract}
\addcontentsline{toc}{subsection}{Abstract}


Mosquitoes are a vector for disease, prevention of disease transmitted in this way is where the motivation of this report lies. Annually, over one million people die from mosquito-borne disease, hundreds of millions are put at risk. The ability to detect the presence of a mosquito within a sound recording is an invaluable tool for use in active warning systems, tracking swarm movements, and data-logging. All of these applications are valuable in understanding and preventing the spread of these diseases. 

The following report establishes an extensive and detailed pipeline for detection of a particular mosquito species, Culex \textit{quinquefasciatus}, against general background noise and extends it to testing against recordings of other species. Alongside the building of a detection pipeline, a software package, '\textit{MozzPy}', has been developed to provide an environment for algorithm testing, discussion of which augments the report throughout.

The database of recordings has been provided by Kew Gardens and XYZ, where the audio capture has been achieved with inexpensive acoustic sensors found within low-end smartphones, enabling large volumes of diverse data to be collected and allowing any resulting detectors to be run on a mobile platform, accessible to anyone with a smartphone.

Novel contributions are made within the general field of machine learning with the combined use of asymmetrical rejection and classifier result aggregation. Resultant performance is very good, with accuracies exceeding the 90\% range subject to varying degrees of rejection, depending on the application.